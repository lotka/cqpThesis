\section{Transfer matrix analysis of ring resonator cavities}
\label{ringResAppen}
A useful way to describe the spectral response of a ring resonator is by defining a transfer matrix. This $2\times2$ matrix relates the electric fields of the inputs and outputs of the waveguide and resonator. In our analysis we define the electric fields to be complex valued and also normalised to conserve energy, this condition implies the transfer matrix must be unitary. 

First we write down relations between different parts of the structure, referencing figure \ref{ringModel}:
% Equation for E_out
\begin{equation}
E_{out}=r_1E_{in}+t_1E_2
\end{equation}
%Equation for E_1
\begin{equation}
E_{1}=r_2E_2+t_2E_{in}
\end{equation}
Define a matrix.
\begin{equation}
M = \begin{pmatrix}
r_1 & t_1 \\
t_2 & r_2 \\
\end{pmatrix}
\end{equation}
Enforce unitarity
\begin{equation}
\begin{pmatrix}
r_1 & t_1 \\
t_2 & r_2 \\
\end{pmatrix}
\begin{pmatrix}
r^*_1 & t^*_2 \\
t^*_1 & r^*_2 \\
\end{pmatrix}
=
\begin{pmatrix}
|r_1|^2+|t_1|^2 & r_1t^*_2+r^*_2t_1 \\
r_2t^*_1+r^*_1t_2 & |r_2|^2+|t_2|^2 \\
\end{pmatrix}=
\begin{pmatrix}
1 & 0 \\
0 & 1 \\
\end{pmatrix}
\end{equation}
Furthermore $\det{M}=1$
\begin{equation}
r_1r_2-t_1t_2=1
\end{equation}
so 
\begin{equation}
r_2=\frac{1+t_1t_2}{r_1}
\end{equation}
Also $M^{-1}=M^*$
\begin{equation}
\begin{pmatrix}
r_2 & -t_1 \\
-t_2 & r_1 \\
\end{pmatrix}=
\begin{pmatrix}
r^*_1 & t^*_1 \\
t^*_2 & r^*_2 \\
\end{pmatrix}
\end{equation}
This tells us $r_1$ and $r_2$ must be real.
So we can simplify the unitarity equation. 
So
\begin{equation}
t_1=iT_1
\end{equation}
\begin{equation}
t_2=iT_2
\end{equation}
Rewrite the interesting equations:
Use also $r_2=r^*_1$
\begin{equation}
\begin{pmatrix}
|r_1|^2+T_1^2 & -ir_1T_2+ir_1T_1 \\
-ir^*_1T_1+ir^*_1T_2 & |r_1|^2+T_2^2 \\
\end{pmatrix}=
\begin{pmatrix}
1 & 0 \\
0 & 1 \\
\end{pmatrix}
\end{equation}
Because $T_2=T_1=t$:
\begin{equation}
\begin{pmatrix}
|r_1|^2+t^2 & 0 \\
0 & |r_1|^2+t^2 \\
\end{pmatrix}=
\begin{pmatrix}
1 & 0 \\
0 & 1 \\
\end{pmatrix}
\end{equation}
\begin{equation}
|r_1|^2=1-t^2
\end{equation}
\begin{equation}
|r_2|^2=1-t^2
\end{equation}d
There is a degree of freedom. $r_1=e^{i\theta}r$ and $r_2=e^{-i\theta}r$. Choose $\theta = 0$ for simplicity.

Now the matrix is:
\begin{equation}
\begin{pmatrix}
r & it \\
it & r \\
\end{pmatrix}
\end{equation}

Adding that $$E_2=\tau e^{i\theta}E_1$$

Recalling the initial equations.
% Equation for E_out
\begin{equation}
E_{out}=rE_{in}+itE_2
\end{equation}
%Equation for E_1
\begin{equation}
E_{1}=rE_2+itE_{in}
\end{equation}
%Equation for E_1
\begin{equation}
E_{1}=r\tau e^{i\theta}E_1+itE_{in}
\end{equation}
%
\begin{equation}
\frac{E_{1}}{E_{0}}=\frac{it}{1-r\tau e^{i\theta}}
\end{equation}
Taking the absolute value gives:
\begin{equation}
\left|\frac{E_{1}}{E_{0}}\right |^2=\frac{t^2}{1+r^2\tau^2-2r\tau cos(\theta)}
\end{equation}
For $E_{2}$ we have:
\begin{equation}
\frac{E_{2}}{E_{0}}=\frac{it\tau e^{i\theta}}{1-r\tau e^{i\theta}}
\end{equation}
%
\begin{equation}
\left |\frac{E_{2}}{E_{0}}\right |^2=\frac{\tau^2-\tau^2r^2}{1+r^2\tau^2-2r\tau cos(\theta)}
\end{equation}
For $E_{out}$ we have:
\begin{equation}
\frac{E_{out}}{E_{0}}=r+it\frac{it\tau e^{i\theta}}{1-r\tau e^{i\theta}}
\end{equation}
\begin{equation}
\frac{E_{out}}{E_{0}}=\frac{r-r^2\tau e^{i\theta}-t^2\tau e^{i\theta}}{1-r\tau e^{i\theta}}
\end{equation}
%abs val
\begin{equation}
\left |\frac{E_{out}}{E_{0}}\right|^2=\frac{(r-r^2\tau e^{i\theta}-t^2\tau e^{i\theta})(r-r^2\tau e^{-i\theta}-t^2\tau e^{-i\theta})}{(1-r\tau e^{i\theta})(1-r\tau e^{-i\theta})}
\end{equation}
\begin{equation}
\left |\frac{E_{out}}{E_{0}}\right|^2=\frac{(r(r-r^2\tau e^{-i\theta}-t^2\tau e^{-i\theta})-r^2\tau e^{i\theta}(r-r^2\tau e^{-i\theta}-t^2\tau e^{-i\theta})-t^2\tau e^{i\theta}(r-r^2\tau e^{-i\theta}-t^2\tau e^{-i\theta}))}{1+r^2\tau^2-2r\tau cos(\theta)}
\end{equation}
%
\begin{equation}
\left |\frac{E_{out}}{E_{0}}\right|^2=\frac{r^2-r^3\tau e^{-i\theta}-rt^2\tau e^{-i\theta}-r^3\tau e^{i\theta}+r^4\tau^2+2r^2\tau^2t^2-rt^2\tau e^{i\theta}+t^4\tau^2}{1+r^2\tau^2-2r\tau cos(\theta)}
\end{equation}
%
\begin{equation}
\left |\frac{E_{out}}{E_{0}}\right|^2=\frac{r^2-e^{-i\theta}(r^3\tau+rt^2\tau)-e^{i\theta}(r^3\tau+rt^2\tau)+r^4\tau^2+t^4\tau^2+2r^2\tau^2t^2}{1+r^2\tau^2-2r\tau cos(\theta)}
\end{equation}
%
\begin{equation}
\left |\frac{E_{out}}{E_{0}}\right|^2=\frac{r^2+2(r^3\tau+rt^2\tau)\cos(\theta)+r^4\tau^2+t^4\tau^2+2r^2\tau^2t^2}{1+r^2\tau^2-2r\tau cos(\theta)}
\end{equation}
%
\begin{equation}
\left |\frac{E_{out}}{E_{0}}\right|^2=\frac{r^22(r^3\tau+r(1-r^2)\tau)\cos(\theta)+r^4\tau^2+(1-r^2)^2\tau^2+2r^2\tau^2(1-r^2)}{1+r^2\tau^2-2r\tau cos(\theta)}
\end{equation}
%
\begin{equation}
\left |\frac{E_{out}}{E_{0}}\right|^2=\frac{r^2-2(r^3\tau+r(1-r^2)\tau)\cos(\theta)+r^4\tau^2+(1-2r^2+r^4)\tau^2+2r^2\tau^2-2r^4\tau^2}{1+r^2\tau^2-2r\tau cos(\theta)}
\end{equation}
%
\begin{equation}
\left |\frac{E_{out}}{E_{0}}\right|^2=\frac{r^2-2r\tau\cos(\theta)+r^4\tau^2+\tau^2-2r^2\tau^2+r^4\tau^2+2r^2\tau^2-2r^4\tau^2}{1+r^2\tau^2-2r\tau cos(\theta)}
\end{equation}
%
\bf{FINAL RESULTS}
\begin{equation}
\left |\frac{E_{out}}{E_{0}}\right|^2=\frac{r^2-2r\tau\cos(\theta)+\tau^2}{1+r^2\tau^2-2r\tau cos(\theta)}
\end{equation}
\begin{equation}
\left |\frac{E_{2}}{E_{0}}\right |^2=\frac{\tau^2-\tau^2r^2}{1+r^2\tau^2-2r\tau cos(\theta)}=\tau^2\left|\frac{E_{1}}{E_{0}}\right |^2
\end{equation}
\begin{equation}
\left|\frac{E_{1}}{E_{0}}\right |^2=\frac{t^2}{1+r^2\tau^2-2r\tau cos(\theta)}
\end{equation}