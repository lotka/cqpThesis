\newpage
\section{Introduction}
If the endeavour to build a quantum computer is successful, computational problems which are currently intractable on classical computers will be solvable. A particularly promising paradigm for this is the linear optical quantum computer (LOQC) model which in theory allows for scalable universal quantum computation. Work on LOQC can be done using bulk optics components but this quickly becomes impractical when the experiments need to be scaled up to more qubits. Integrated photonics is the solution to this problem and allows for experiments with a much higher component density, an essential ingredient to scalable quantum computing. Many components from bulk optics have equivalents on these photonic chips with popular materials being silicon-on-insulator (SOI) , lithium niobate and glass materials. Here we focus on SOI chips as they have many promising properties for the implementation of complex quantum optical circuits.

A key requirement for the full implementation of a LOQC is a scalable, bright, deterministic and indistinguishable single photon source. Single photon sources in the SOI platform are typically made from the waveguide itself and use spontaneous four-wave mixing which occurs in silicon due to a third order non-linearity to create a single photon pair. This report aims to develop a method of measuring the indistinguishably of the produced photons with a classical technique, exploiting stimulated four-wave mixing. This method collects a joint spectrum which is an estimation of the spectral shape of the two photons produced by the source. For a full description, the Joint Spectral Amplitude JSA is the desired quantity and this is a full description of the wavefunction of the single photons emitted by the silicon ring resonators. However it is only within the scope of this work to measure the Joint Spectral Intensity, which is the absolute value squared of the JSA. This gives an estimate of the JSA and allows the purity to be bounded from above.

The mission is therefore to develop a methodology to reconstruct these wavefunctions and hence engineer indistinguishable (high purity) single photon sources. In this work we performed such measurements on two SOI chips. The experimental work started with an initial proof of concept that one can collect joint spectrum data in the way desired. This was done on a chip supplied by Marc Sorel from Glassgow University. Finally in order to investigate a promising new material,an amorphous silicon chip from Hewlett Packard was used for experiments. 

In parallel techniques of analysing the output data are developed. Filtering techniques which remove noise are tested in order to extract the relevant signals. A general framework is set out which aims to quantify the uncertainty in the measurements.

Finally we conclude that there is still much to be done in this area, proposing an outline for how to carry out effective measurements in the future.


