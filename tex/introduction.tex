\newpage
\section{Introduction}
The endeavour to build a quantum computer holds the promise of solving computational problems which are currently intractable on classical computers. A particularly promising paradigm for this is the linear optical quantum computer (LOQC) model which in theory allows for scalable universal quantum computation. Work on LOQC can be done using bulk optics components but this quickly becomes impractical when the experiments need to be scaled up to more qubits. Integrated photonics is a solution to this problem and allows for experiments with more qubits in a much smaller space. Optical circuits can be implemented on such chips, popular materials are silicon-on-insulator (SOI) , lithium niobate and glass materials. Here we focus on SOI chips as they have many promising properties for the implementation of complex quantum optical circuits.

A key requirement for the full implementation of LOQC is a scalable, bright, deterministic and indistinguishable single photon source. Single photon sources in the SOI platform are typically made from the waveguide itself and use the spontaneous four-wave mixing which occurs in silicon due to the third order non-linearity, creating a single photon pair. This report aims to develop a method of measuring the indistinguishably of the produced photons with a classical technique, exploiting stimulated four-wave mixing. This method collects a joint spectrum which is an estimation of the spectral shape of the two photons produced by the source. For a full description the Joint Spectral Amplitude JSA is the desired quantity, this is a full description of the wavefunction our the single photons emitted by the silicon ring resonators. However it is only within the scope of this work to measure the Joint Spectral Intensity, which is the absolute value squared the JSA. This allow

The mission is therefore to develop a methodology to reconstruct these wavefunctions and hence engineer indistinguishable (high purity) single photon sources. In this work we performed such measurements on three SOI chips. The experimental work started with an initial proof of concept that one can collect joint spectrum data in the way desired. This was done on a chip supplied by Marc Sorel from Glassgow University. Then due to the fragility of these chip at high powers the experiment progressed to a chip manufactured by Toshiba. Finally in order to investigate a promising new material amorphous silicon chip was used for experiments. 

In parallel techniques of analysing the output data are developed. Filtering techniques which remove noise are developed in order to make the data usable. A general framework is set out which aims to quantify the certainty in the measurements.

Finally we conclude that there is still much to be done in this area, proposing an outline for how to carry out effective measurements in the future.


