\section{Introduction}
The endeavour to build a quantum computer holds the promise of solving computational problems which are currently intractable on classical computers. A particularly promising paradigm for this is the linear optical quantum computer (LOQC) model which in theory allows for scalable universal quantum computation. Work on LOQC can be done using bulk optics components but this quickly becomes impractical when the experiments needed to be scaled up to more qubits. Integrated photonics has solved this problem and allowed for experiments with more qubits in a much smaller space. Optical circuits can be implemented on such chips, popular materials are silicon-on-insulator (SOI) , lithium niobate and glass materials. Here we focus on SOI chips as they have many promising properties for the implementation of complex optical circuits.

A key requirement for full implementation of LOQC is a scalable, bright, deterministic and indistinguishable single photon source. Single photon sources in the SOI platform are typically made from the waveguide itself and use the spontaneous four-wave mixing which occurs in silicon due to the third order non-linearity. This report aims to develop a new method of measuring the indistinguishably of the produced photons with a classical technique, exploiting stimulated four-wave mixing. This method collects a Joint Spectrum which is an estimation of the wave function of the produced single photon. 

